\section{Neimhin's summaries}
\subsection{Recognition of Affect, Judgement, and Appreciation in Text}
\cite{neviarouskaya-2010}: The authors develop a rule-based algorithm for recognizing attitude in terms of affect (negative and positive), judgement (negative and positive), and appreciation. The algorithm systematically composes the classifications of constituent sentence parts, such that entire sentences can be classified. Their theory and algorithm applies at three levels. At the top level are just three classifications, positive, negative and neutral. At the mid level, positive and negative classifications are further subdivided into, affect, judgement, and appreciation. At the most fine-grained level affect is divided into interest, joy, and surprise (positive affect) and anger, disgust, fear, guilt, sadness, shame (negative affect). The algorithm accounts for negation by several means, amplification of sentiment, neutralisation of sentiment etc. A corpus of human-annotated sentences is collected and compared to the results of the automated system. The automated system was found to have greater agreement with the human annotations than a baseline system that simply selected the most intense token annotation. The authors discuss some of the failures and failure modes of their system. The accuracy of the automated systems (at predicting human annotations) is highest at the top level of the classification hierarchy (positive, negative, neutral), and the accuracy decreases as we move to finer grained classifications.

- If the algorithm proposed in the paper is available as code it would be useful for our textual analyses, but it would too complex to implement from scratch given the time available.

- If an implementation exists it may be possible to make adjustments leveraging other techniques, for instance using word embeddings from a large language model like BERT to make atomic phrasal classifications, and then continuing the authors' algorithm.

- It may be generally true that finer grained analyses/classifications are more difficult than coarser analyses, so the hierarchical approach is useful for validating the methodology at different levels.

- If the corpus described in the paper is available it could be useful to our project.

\subsection{miller-2020}
\cite{miller-2020}: Miller (2020) reports a survey of 628 Reddit users. The questionnaire covered Internet usage, Reddit usage (especially in relation to self-disclosure on subreddits), and a set of psychological questions relating to connectedness, sociol support perception, life satisfaction, narcissism, and sensation seeking.

The sharing of intimate information on Reddit's anonymous-feeling platform was found to be significantly correlated with the sensation seeking tendency, but there was no evidence of an association with narcissicism.

The findings are interpreted as evidence that intimate self-disclosure on Reddit is motivated by feeling disconnected and alone.

The research participants were approached with advertisements on Reddit, as well as with posts to related subreddits, e.g. `r/confessions`, `r/offmychest`.

Miller (2020) suggests that "[f]uture work should investigate whether engagement is driven by empathetic concern, or whether it is spurred by boredom and the tendency of confessions to have a high degree of shock value", which is an interesting line of investigation in the context of spontaneous online advice-seeking. What drives engagement with advice-seeking posts, i.e. what drives Redditors to give advice?


\subsection{pennebaker-2015}
\cite{pennebaker-2015}:
The LIWC2015 program processes texts and produces
linguistic and textual measures, as well as proxy measures
of various psychological and developmental attributes.

The proxy measures are essentially relative word counts.
For instance the 'negative emotion' proxy is calculated by
counting occurences of words associated with negative emotion.

The 'negative emotion' proxy is subdivided into anxiety, anger, and sadness.
Words which contribute to any of these also contribute to the 'negative emotion' score of the text.
Interestingly 'positive emotion' is not subdivided, meaning there is a richer
analysis of negative emotion. I also saw in the paper "Recognition of Affect, Judgement, and Appreciation in Text"
that negative affect was given a richer subcaterogization that positive affect.

The authors describe the process by which the LIWC2015 dictionaries and word-categorisations were developed. It involved a significant manual process with collaboration of several judges to create an initial dictionary. The dictionary was then analysed and ammended to fix ommissions or internal inconsistencies. The set of categories was chosen based on attributes commonly study in social, health, and personality psychology. The authors give an explanation as to why the internal consistency of linguistic-based psychometrics can be lower or have wider margins than a questionnaire (it comes down to the fact that people rarely repeat themselves in spontaneous natural language, whereas questionnaires make heavy use of repetition, or at least rhyming). Perhaps of interest to our group's analysis is the claim that Spearman-Brown prediction formula is generally are more accurate approximation of a word-category's internal consistency than is Cronbach's alpha.
