\section{Research Questions}

\citet[56]{miller-2020} suggested investigating the motivations behind engagement with confessionary public social media postings, in particular whether it is empathetic concern or sensationalism/``shock value'' that drives engagement.
We seek to answer similar questions but in a slightly different forum,
specifically in public spontaneous
advice-seeking discussions with a confessionary bent
on social media. Also, we take in an interest in different types
of engagement, empathetic/compassionate/productive engagement,
in contrast with sensationalist/antagonistic/trolling engagement.

We narrow our focus to discussions initiated by advice-seeking posts with an authentic,
vulnerable tone, avoiding, as far as possible,
the inclusion of deceptive, purely comedic, or mocking submissions
(although {\em responses} with comedic/deceptive/mocking intention are still of interest to us).
In other words, we want to answer questions about cases where
posters are facing a genuine moral dilemma and soliciting advice
in good faith.

Especially interesting to us are the cases where advice-seekers
are given `bad news' (in particular, a negative moral judgement of the advice-seeker's character or behaviour), and yet show appreciation towards
the person who delivers the `bad news'.
What are the textual markers and schematics of such well-received,
negative advice-giving comments?

On the other side of the spectrum, we take an interest
in cases where the public discussion devolves into
antagonistic and hateful rhetoric.
What are the textual markers and schematics advice-seeking
posts that coincide with threads developing poorly?
In other words, can we extract specific words, phrases,
patterns of discourse that one should avoid when genuinely
seeking compassionate and nuanced moral advice
on public social media fora?
Therefore, to answer this question, we are interested
in cases in which the original advice-seeking post
exhibits an authentic solicitation relating to
a genuine moral dilemma.

Finally, can there be a therapeutic benefit to the advice-seeker
by submitting their moral dilemma to public scrutiny?
